\documentclass{article}
% Útgáfa 1.5

% pakkar fyrir töflur; array er notað í "math-mode"; arydshln fyrir brotalínur í töflum
\usepackage{array,tabularx}%,arydshln}  
% íslenskt letur, orðaskiptingar ...
\usepackage[english]{babel}
\usepackage[T1]{fontenc}
% númera jöfnur, töflur, myndir, ...
\usepackage{enumerate}
% fyrir krækjur
\usepackage[colorlinks,linkcolor=blue,citecolor=blue,urlcolor=blue]{hyperref}
% ýmis tákn, leturgerðir, ... ATH amsmath fyrir \text{} skipunina
\usepackage{amsmath,amssymb,euscript}   
% til að setja inn *.eps myndir ef við notum dvi og ps skjöl...
\usepackage{epsfig}    % \epsfig{...}
% ... en ef við notum pdf beint, þá er þetta pakkinn sem við þurfum
\usepackage{graphicx}  % \includegraphics[...]{...}
% litamöguleikar fyrir texta
\usepackage{color}
% til þess að myndir og töflur standi þar sem þær eiga að standa!
\usepackage{here}
%
\setlength{\textwidth}{6in}
\setlength{\textheight}{9in}
\setlength{\headheight}{0in}
\setlength{\headsep}{6pt}
\setlength{\topskip}{0in}
\setlength{\topmargin}{0cm}
\setlength{\oddsidemargin}{0in}
% dregur fyrstu línu í hverri málsgrein inn - nota 0cm fyrir engan inndrátt fyrir allar málsgreinar en \noindent í upphafi málsgreinar til að hafa engan inndrátt í þeirri málsgrein einni:
\setlength{\parindent}{0cm}
% viljum hafa eitt línubil milli efnisgreina
\setlength{\parskip}{1.5ex plus 0.75ex minus 0.5ex}
% 1.5 í línubil
\renewcommand{\baselinestretch}{1.025}

% Environments
\newenvironment{alist}[1][$\quad\,$1.]{
\vspace*{-8pt} \begin{enumerate}[label=\alph*),itemsep=4pt,parsep=3pt]}
{\end{enumerate}\vspace*{-3pt}}

\newenvironment{ttafla}[1][$\quad\,$1.]{
\begin{tabular}{rcl} \renewcommand\arraystretch{2} }
{\end{tabular}}



%Bætt við af mér (Hannesi)
% Þjappa saman
\widowpenalty=600
\clubpenalty=600
\usepackage[compact]{titlesec}
\titlespacing{\section}{0pt}{2ex plus 0.75ex minus 0.75ex}{1ex plus 0.3ex minus 0.3ex}
\titlespacing{\subsection}{0pt}{1ex plus 0.25ex minus 0.25ex}{0ex plus 0.1ex minus 0.1ex}
\titlespacing{\subsubsection}{0pt}{0.5ex plus 0.1ex minus 0.1ex}{0ex plus 0.1ex minus 0.1ex}

% Bolda caption
\usepackage[hang,small,bf]{caption}

% Efnaformúlur og myndir
%\usepackage[version=3]{mhchem}
\usepackage{chemfig}

% Komma í stað punkts
\usepackage{icomma}

% Annað
\usepackage{subfig}
\usepackage{array}
\usepackage{bigstrut}
\usepackage{multirow}
\usepackage{multicol}
\usepackage{enumerate}
%\usepackage{enumitem}
\usepackage{wrapfig}
\newcommand{\HRule}{\rule{\linewidth}{0.5mm}}
\usepackage{ulem}
%\usepackage[thinspace,mediumqspace,amssymb]{SIunits}

%Tikz
\usepackage{tikz}
\usetikzlibrary{fit,arrows,decorations.pathmorphing,decorations.text,backgrounds,positioning,fit,petri,3d,calc}
%% 3d hnitakerfi
\usepackage{tikz-3dplot}
\tdplotsetmaincoords{75}{130}
%\begin{tikzpicture}[scale=5,tdplot_main_coords]
%
%%set up some coordinates 
%%-----------------------
%\coordinate (O) at (0,0,0);
%
%%determine a coordinate (P) using (r,\theta,\phi) coordinates.  This command
%%also determines (Pxy), (Pxz), and (Pyz): the xy-, xz-, and yz-projections
%%of the point (P).
%%syntax: \tdplotsetcoord{Coordinate name without parentheses}{r}{\theta}{\phi}
%\tdplotsetcoord{P}{\rvec}{\thetavec}{\phivec}


%%%%%%%%%%%%%%% SKIPANIR %%%%%%%%%%%%%%%%%%%%%%%%
%% LITASTYTTINGAR
%
\definecolor{dgreen}{rgb}{0,0.8,0}
\newcommand{\red}[1]{{\color{red} #1}}
\newcommand{\green}[1]{{\color{dgreen} #1}}
\newcommand{\blue}[1]{{\color{blue} #1}}
\newcommand{\black}[1]{{\color{black} #1}}
%
% ALLS KYNS STÆRÐFRÆÐIDÓT
%
\newcommand{\lvec}{\overrightarrow}
%
% ýmsar diffurstyttingar
\newcommand{\dif}{\,\mathrm{d}}
\newcommand{\dx}{\,\mathrm{d}x}
\newcommand{\dy}{\,\mathrm{d}y}
\newcommand{\dz}{\,\mathrm{d}z}
\newcommand{\dt}{\,\mathrm{d}t}
% afleiður og hlutafleiður
\renewcommand{\d}[2]{\frac{\dif{#1}}{\dif{#2}}}
\newcommand{\dd}[3]{\frac{\dif^{#1}{#2}}{\dif^{#1}{#3}}}
\newcommand{\p}[2]{\frac{\partial{#1}}{\partial {#2}}}
\newcommand{\pp}[3]{\frac{\partial^{#1}{#2}}{\partial{#3}^{#1}}}
%
% bil, millistig \; og \quad
\newcommand{\bil}{\hspace*{6pt}}
\newcommand{\Bil}{\hspace*{10pt}}
\newcommand{\vbil}{\vspace*{-6pt}}
\newcommand{\vBil}{\vspace*{-10pt}}

% samasemmerki með aukaplássi á báða bóga
\newcommand{\bils}{\bil=\bil}
%
\newcommand{\bc}{\begin{center}}
\newcommand{\ec}{\end{center}}
%
\newcommand{\beq}{\begin{equation}}
\newcommand{\eeq}{\end{equation}}
%
\newcommand{\beqa}{\begin{eqnarray*}}
\newcommand{\eeqa}{\end{eqnarray*}}
%
\newcommand{\ba}{\begin{array}}
\newcommand{\ea}{\end{array}}
%
\newcommand{\bma}{\begin{matrix}}
\newcommand{\ema}{\end{matrix}}
%
\newcommand{\bmh}{\left[\begin{matrix}}
\newcommand{\emh}{\end{matrix}\right]}
%
\newcommand{\ts}{\textstyle}
\newcommand{\ds}{\displaystyle}

% Frá mér (Hannesi)
\newcommand{\lausn}{\textbf{\textit{Lausn:}}}
\newcommand{\solution}{\textbf{\textit{Solution:}}}
\newcommand{\daemi}{\textbf{\textit{Dæmi:}}}
\newcommand{\bt}{\begin{table}[H]}
\newcommand{\et}{\end{table}}
\newcommand{\atm}{\text{atm}}
\newcommand{\calories}{\text{cal}}
\everymath{\displaystyle} % Svo allar jöfnur taka aldrei minna pláss
\renewcommand\tabcolsep{2pt} % Bil dalka í töflum
\setlength{\arraycolsep}{1.5pt}
%\usepackage{fancyhdr}
%\setlength{\headheight}{25pt}
%\pagestyle{fancy}
%\renewcommand{\headrulewidth}{0.4pt}
%\renewcommand{\footrulewidth}{0.4pt}
%\usepackage[top=2.5in, bottom=1.5in, left=1in, right=1in]{geometry}
\usepackage{amssymb}
\usepackage{pifont}
\newcommand{\cmark}{\text{\ding{51}}}%
\newcommand{\xmark}{\text{\ding{55}}}%

\usepackage{pbox}
%\usepackage{fancyvrb}
\usepackage{array}
\newcolumntype{L}[1]{>{\raggedright\let\newline\\\arraybackslash\hspace{0pt}}m{#1}}
\newcolumntype{C}[1]{>{\centering\let\newline\\\arraybackslash\hspace{0pt}}m{#1}}
\newcolumntype{R}[1]{>{\raggedleft\let\newline\\\arraybackslash\hspace{0pt}}m{#1}}
\usepackage[utf8]{inputenc}
%\lhead{Þróun hugbúnaðar \\ Group assignment 1 - Group: F2a}
%\rhead{Teachers: Matthias Book \& Kristín Fjóla Tómasdóttir \\ Students: Einar Helgi, Hannes Pétur \& Sigurður Birkir}
\usepackage{framed}

%\begin{ustory}{title}{duration}{priority}
% 1 or more \task
%\end{ustory}
\newenvironment{ustory}[3]
{ \begin{minipage}{0.495\textwidth} \begin{framed} \rule{1ex}{1ex} #1 \hspace{\stretch{1}} \rule{1ex}{1ex} \raisebox{0.3cm}{\rule{1\textwidth}{0.4pt}} \begin{flushright}\vspace*{-0.65cm}\textbf{#2, Priority: #3} \end{flushright} \textbf{\textit{Tasks:}} \begin{enumerate}[a)]}
{\end{enumerate} \end{framed} \end{minipage}}
\definecolor{altgreen}{rgb}{0.1, 0.55, 0.2}

\newcommand\xput[2][0.5]{%
    \rule{#1\textwidth}{0pt}\makebox[0pt][c]{#2}\hfill}

%\task[duration]{task}{assigned to who?}
%
% Case 0: ???
% Case 1: Einar
% Case 2: Hannes
% Case 3: Sigurður
\newcommand{\task}[3][? days]{
	\ifcase#3
		\item #2 \\ \textbf{             {\color{red} }} \hfill\textbf{(#1)}
	\or
		\item #2 \\ \textbf{Assigned to: {\color{red} Einar}} \hfill \textbf{(#1)}
	\or
		\item #2 \\ \textbf{Assigned to: {\color{altgreen} Hannes}} \hfill \textbf{(#1)}
	\or
		\item #2 \\ \textbf{Assigned to: {\color{blue} Sigurður}} \hfill \textbf{(#1)}
	\fi
	\vspace*{0.1cm}
}

\usepackage{paralist}
\renewenvironment{enumerate}[1]{\begin{compactenum}#1}{\end{compactenum}}
\usetikzlibrary{shapes.multipart,positioning}

\begin{document}
\begin{titlepage}
\begin{center}

\textsc{}\\[2cm] 

\includegraphics[width=6cm]{Haskoli_Islands_rett.jpg}\\[0.5cm]

\HRule \\[0.6cm]
{ \huge \bfseries Group assignment 2: }\\[0.2cm]
\HRule \\[0.4cm]

\textsc{\normalsize Þróun hugbúnaðar} \\
\textsc{Spring 2015} \\[1.5cm]

\begin{minipage}{0.45\textwidth}
\begin{flushleft} \large
\textit{Students:} (Group F2a)\\
\textsc{Einar Helgi Þrastarson} \\
\textsc{Hannes Pétur Eggertsson} \\
\textsc{Sigurður Birkir Sigurðsson} \\
\end{flushleft}
\end{minipage}
\begin{minipage}{0.45\textwidth}
\begin{flushright} \large
\textit{Teachers:} \\
\textsc{Matthias Book}\\
\textsc{Kristín Fjóla Tómasdóttir}\\
\textsc{ }\\
\end{flushright}
\end{minipage}

\end{center}
\end{titlepage}


% Description of the project
\section{Class diagram}
In this document there's the class diagram for group F2a. Group members are: Einar Helgi Þrastarson (personal ID number: 110287-2919), Hannes Pétur Eggertsson (240889-2939) and Sigurður Birkir Sigurðsson (120589-2539). Our project is to build an user interface for a fantasy football game. In our class diagram we felt it made sense to split the classes into two categories, back-end classes and front-end classes.

In our class diagrams we use the following notation:\vspace*{-0.5cm}
\begin{itemize}\itemsep-4pt
\item[--] means a private variable or method (not directly accessable by other classed).
\item[+] means a public variable or method (directly accessable by other methods).
\end{itemize}

\tikzset{
    umlclass/.style={
        draw=black,fill=yellow!16,rectangle split,align=center, rectangle split part align={center,left}, minimum width=5cm,rounded corners},draw,rectangle split parts=4}

Also, each class in the diagram has four sections:
\begin{center}


\begin{tikzpicture}
\begin{scope}[xshift=0cm,yshift=0cm]
\node[umlclass] (t1)
 {\textbf{\large \textit{The class' name.}}
 \nodepart{two}
  {\footnotesize
  \begin{tabular}{l} 
   Short description of the class.
  \end{tabular}}
 \nodepart{three}
  \begin{tabular}{l}
  The class' variables and their type listed\\
  on the format:\\
   --/+ \texttt{type variable}\\
  \end{tabular}
 \nodepart{four}
  \begin{tabular}{l}
   The class' methods listed on the format:\\
   --/+ \texttt{void changeScore(int dScore)}\\
  \end{tabular}
 };
\end{scope}
\end{tikzpicture}
\end{center}



\newpage
\section{Class diagram}
\begin{tikzpicture}
\begin{scope}[xshift=0cm,yshift=0cm]
\node[umlclass] (t1)
 {\textbf{\large \textit{User}}
 \nodepart{two}
  {\footnotesize
  \begin{tabular}{l} 
   This class keeps track of all information\\
   about each user playing the game.\\
   
  \end{tabular}}
 \nodepart{three}
  \begin{tabular}{l}
   -- \texttt{int money}\\
   -- \texttt{int score}\\
   -- \texttt{String name}\\
   -- \texttt{Roster roster}\\
   -- \texttt{Scoreboard scoreB}
  \end{tabular}
 \nodepart{four}
  \begin{tabular}{l}
   + \texttt{changeMoney(int dMoney)}\\
   + \texttt{changeScore(int dScore)}\\
  \end{tabular}
 };
\end{scope}
  %\draw (t1.text split) -- (t1.two split);

  \begin{scope}[xshift=7cm]
    \node[rectangle split, rectangle split parts=5,
          draw, minimum width=4cm,font=\small,
         rectangle split part align={center}] (t2)
     {                \textbf{Name : glyph}
       \nodepart{two}
                      prev \hspace*{4ex} next
       \nodepart{three}
                      width: 15pt  
       \nodepart{four}
                      height: 15pt 
       \nodepart{five}
                      list}; 
    \draw (t2.text split) -- (t2.two split);
  \end{scope}

  \draw[arrows={angle 60-latex},thick] (t1.three east) to [out=0, in=170](t2.text west) node[above left]{1};
  \draw[arrows={angle 60-latex},thick] ([yshift=10pt] t1.three east) to [out=0, in=170](t2.two west) node[above left]{1};
\end{tikzpicture}

\newpage
\section*{Appendix}
\subsection*{User interface}


\end{document}