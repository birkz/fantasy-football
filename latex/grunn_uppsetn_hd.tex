% Útgáfa 1.5

% pakkar fyrir töflur; array er notað í "math-mode"; arydshln fyrir brotalínur í töflum
\usepackage{array,tabularx}%,arydshln}  
% íslenskt letur, orðaskiptingar ...
\usepackage[english]{babel}
\usepackage[T1]{fontenc}
% númera jöfnur, töflur, myndir, ...
\usepackage{enumerate}
% fyrir krækjur
\usepackage[colorlinks,linkcolor=blue,citecolor=blue,urlcolor=blue]{hyperref}
% ýmis tákn, leturgerðir, ... ATH amsmath fyrir \text{} skipunina
\usepackage{amsmath,amssymb,euscript}   
% til að setja inn *.eps myndir ef við notum dvi og ps skjöl...
\usepackage{epsfig}    % \epsfig{...}
% ... en ef við notum pdf beint, þá er þetta pakkinn sem við þurfum
\usepackage{graphicx}  % \includegraphics[...]{...}
% litamöguleikar fyrir texta
\usepackage{color}
% til þess að myndir og töflur standi þar sem þær eiga að standa!
\usepackage{here}
%
\setlength{\textwidth}{7in}
\setlength{\textheight}{9in}
\setlength{\headheight}{0in}
\setlength{\headsep}{6pt}
\setlength{\topskip}{0in}
\setlength{\topmargin}{0cm}
\setlength{\oddsidemargin}{-0.3in}
\setlength{\marginparwidth}{10pt}
% dregur fyrstu línu í hverri málsgrein inn - nota 0cm fyrir engan inndrátt fyrir allar málsgreinar en \noindent í upphafi málsgreinar til að hafa engan inndrátt í þeirri málsgrein einni:
\setlength{\parindent}{0cm}
% viljum hafa eitt línubil milli efnisgreina
\setlength{\parskip}{1.5ex plus 0.75ex minus 0.5ex}
% 1.5 í línubil
\renewcommand{\baselinestretch}{1.025}

% Environments
\newenvironment{alist}[1][$\quad\,$1.]{
\vspace*{-8pt} \begin{enumerate}[label=\alph*),itemsep=4pt,parsep=3pt]}
{\end{enumerate}\vspace*{-3pt}}

\newenvironment{ttafla}[1][$\quad\,$1.]{
\begin{tabular}{rcl} \renewcommand\arraystretch{2} }
{\end{tabular}}



%Bætt við af mér (Hannesi)
% Þjappa saman
\widowpenalty=600
\clubpenalty=600
\usepackage[compact]{titlesec}
\titlespacing{\section}{0pt}{2ex plus 0.75ex minus 0.75ex}{1ex plus 0.3ex minus 0.3ex}
\titlespacing{\subsection}{0pt}{1ex plus 0.25ex minus 0.25ex}{0ex plus 0.1ex minus 0.1ex}
\titlespacing{\subsubsection}{0pt}{0.5ex plus 0.1ex minus 0.1ex}{0ex plus 0.1ex minus 0.1ex}

% Bolda caption
\usepackage[hang,small,bf]{caption}

% Efnaformúlur og myndir
%\usepackage[version=3]{mhchem}
\usepackage{chemfig}

% Komma í stað punkts
\usepackage{icomma}

% Annað
\usepackage{subfig}
\usepackage{array}
\usepackage{bigstrut}
\usepackage{multirow}
\usepackage{multicol}
\usepackage{enumerate}
%\usepackage{enumitem}
\usepackage{wrapfig}
\newcommand{\HRule}{\rule{\linewidth}{0.5mm}}
\usepackage{ulem}
%\usepackage[thinspace,mediumqspace,amssymb]{SIunits}

%Tikz
\usepackage{tikz}
\usetikzlibrary{fit,arrows,decorations.pathmorphing,decorations.text,backgrounds,positioning,fit,petri,3d,calc}
%% 3d hnitakerfi
\usepackage{tikz-3dplot}
\tdplotsetmaincoords{75}{130}
%\begin{tikzpicture}[scale=5,tdplot_main_coords]
%
%%set up some coordinates 
%%-----------------------
%\coordinate (O) at (0,0,0);
%
%%determine a coordinate (P) using (r,\theta,\phi) coordinates.  This command
%%also determines (Pxy), (Pxz), and (Pyz): the xy-, xz-, and yz-projections
%%of the point (P).
%%syntax: \tdplotsetcoord{Coordinate name without parentheses}{r}{\theta}{\phi}
%\tdplotsetcoord{P}{\rvec}{\thetavec}{\phivec}


%%%%%%%%%%%%%%% SKIPANIR %%%%%%%%%%%%%%%%%%%%%%%%
%% LITASTYTTINGAR
%
\definecolor{dgreen}{rgb}{0,0.8,0}
\newcommand{\red}[1]{{\color{red} #1}}
\newcommand{\green}[1]{{\color{dgreen} #1}}
\newcommand{\blue}[1]{{\color{blue} #1}}
\newcommand{\black}[1]{{\color{black} #1}}
%
% ALLS KYNS STÆRÐFRÆÐIDÓT
%
\newcommand{\lvec}{\overrightarrow}
%
% ýmsar diffurstyttingar
\newcommand{\dif}{\,\mathrm{d}}
\newcommand{\dx}{\,\mathrm{d}x}
\newcommand{\dy}{\,\mathrm{d}y}
\newcommand{\dz}{\,\mathrm{d}z}
\newcommand{\dt}{\,\mathrm{d}t}
% afleiður og hlutafleiður
\renewcommand{\d}[2]{\frac{\dif{#1}}{\dif{#2}}}
\newcommand{\dd}[3]{\frac{\dif^{#1}{#2}}{\dif^{#1}{#3}}}
\newcommand{\p}[2]{\frac{\partial{#1}}{\partial {#2}}}
\newcommand{\pp}[3]{\frac{\partial^{#1}{#2}}{\partial{#3}^{#1}}}
%
% bil, millistig \; og \quad
\newcommand{\bil}{\hspace*{6pt}}
\newcommand{\Bil}{\hspace*{10pt}}
\newcommand{\vbil}{\vspace*{-6pt}}
\newcommand{\vBil}{\vspace*{-10pt}}

% samasemmerki með aukaplássi á báða bóga
\newcommand{\bils}{\bil=\bil}
%
\newcommand{\bc}{\begin{center}}
\newcommand{\ec}{\end{center}}
%
\newcommand{\beq}{\begin{equation}}
\newcommand{\eeq}{\end{equation}}
%
\newcommand{\beqa}{\begin{eqnarray*}}
\newcommand{\eeqa}{\end{eqnarray*}}
%
\newcommand{\ba}{\begin{array}}
\newcommand{\ea}{\end{array}}
%
\newcommand{\bma}{\begin{matrix}}
\newcommand{\ema}{\end{matrix}}
%
\newcommand{\bmh}{\left[\begin{matrix}}
\newcommand{\emh}{\end{matrix}\right]}
%
\newcommand{\ts}{\textstyle}
\newcommand{\ds}{\displaystyle}

% Frá mér (Hannesi)
\newcommand{\lausn}{\textbf{\textit{Lausn:}}}
\newcommand{\solution}{\textbf{\textit{Solution:}}}
\newcommand{\daemi}{\textbf{\textit{Dæmi:}}}
\newcommand{\bt}{\begin{table}[H]}
\newcommand{\et}{\end{table}}
\newcommand{\atm}{\text{atm}}
\newcommand{\calories}{\text{cal}}
\everymath{\displaystyle} % Svo allar jöfnur taka aldrei minna pláss
\renewcommand\tabcolsep{2pt} % Bil dalka í töflum
\setlength{\arraycolsep}{1.5pt}
%\usepackage{fancyhdr}
%\setlength{\headheight}{25pt}
%\pagestyle{fancy}
%\renewcommand{\headrulewidth}{0.4pt}
%\renewcommand{\footrulewidth}{0.4pt}
%\usepackage[top=2.5in, bottom=1.5in, left=1in, right=1in]{geometry}
\usepackage{amssymb}
\usepackage{pifont}
\newcommand{\cmark}{\text{\ding{51}}}%
\newcommand{\xmark}{\text{\ding{55}}}%

\usepackage{pbox}
%\usepackage{fancyvrb}
\usepackage{array}
\newcolumntype{L}[1]{>{\raggedright\let\newline\\\arraybackslash\hspace{0pt}}m{#1}}
\newcolumntype{C}[1]{>{\centering\let\newline\\\arraybackslash\hspace{0pt}}m{#1}}
\newcolumntype{R}[1]{>{\raggedleft\let\newline\\\arraybackslash\hspace{0pt}}m{#1}}