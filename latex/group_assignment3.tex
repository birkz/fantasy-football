\documentclass{article}
% Útgáfa 1.5

% pakkar fyrir töflur; array er notað í "math-mode"; arydshln fyrir brotalínur í töflum
\usepackage{array,tabularx}%,arydshln}  
% íslenskt letur, orðaskiptingar ...
\usepackage[english]{babel}
\usepackage[T1]{fontenc}
% númera jöfnur, töflur, myndir, ...
\usepackage{enumerate}
% fyrir krækjur
\usepackage[colorlinks,linkcolor=blue,citecolor=blue,urlcolor=blue]{hyperref}
% ýmis tákn, leturgerðir, ... ATH amsmath fyrir \text{} skipunina
\usepackage{amsmath,amssymb,euscript}   
% til að setja inn *.eps myndir ef við notum dvi og ps skjöl...
\usepackage{epsfig}    % \epsfig{...}
% ... en ef við notum pdf beint, þá er þetta pakkinn sem við þurfum
\usepackage{graphicx}  % \includegraphics[...]{...}
% litamöguleikar fyrir texta
\usepackage{color}
% til þess að myndir og töflur standi þar sem þær eiga að standa!
\usepackage{here}
%
\setlength{\textwidth}{6in}
\setlength{\textheight}{9in}
\setlength{\headheight}{0in}
\setlength{\headsep}{6pt}
\setlength{\topskip}{0in}
\setlength{\topmargin}{0cm}
\setlength{\oddsidemargin}{0in}
% dregur fyrstu línu í hverri málsgrein inn - nota 0cm fyrir engan inndrátt fyrir allar málsgreinar en \noindent í upphafi málsgreinar til að hafa engan inndrátt í þeirri málsgrein einni:
\setlength{\parindent}{0cm}
% viljum hafa eitt línubil milli efnisgreina
\setlength{\parskip}{1.5ex plus 0.75ex minus 0.5ex}
% 1.5 í línubil
\renewcommand{\baselinestretch}{1.025}

% Environments
\newenvironment{alist}[1][$\quad\,$1.]{
\vspace*{-8pt} \begin{enumerate}[label=\alph*),itemsep=4pt,parsep=3pt]}
{\end{enumerate}\vspace*{-3pt}}

\newenvironment{ttafla}[1][$\quad\,$1.]{
\begin{tabular}{rcl} \renewcommand\arraystretch{2} }
{\end{tabular}}



%Bætt við af mér (Hannesi)
% Þjappa saman
\widowpenalty=600
\clubpenalty=600
\usepackage[compact]{titlesec}
\titlespacing{\section}{0pt}{2ex plus 0.75ex minus 0.75ex}{1ex plus 0.3ex minus 0.3ex}
\titlespacing{\subsection}{0pt}{1ex plus 0.25ex minus 0.25ex}{0ex plus 0.1ex minus 0.1ex}
\titlespacing{\subsubsection}{0pt}{0.5ex plus 0.1ex minus 0.1ex}{0ex plus 0.1ex minus 0.1ex}

% Bolda caption
\usepackage[hang,small,bf]{caption}

% Efnaformúlur og myndir
%\usepackage[version=3]{mhchem}
\usepackage{chemfig}

% Komma í stað punkts
\usepackage{icomma}

% Annað
\usepackage{subfig}
\usepackage{array}
\usepackage{bigstrut}
\usepackage{multirow}
\usepackage{multicol}
\usepackage{enumerate}
%\usepackage{enumitem}
\usepackage{wrapfig}
\newcommand{\HRule}{\rule{\linewidth}{0.5mm}}
\usepackage{ulem}
%\usepackage[thinspace,mediumqspace,amssymb]{SIunits}

%Tikz
\usepackage{tikz}
\usetikzlibrary{fit,arrows,decorations.pathmorphing,decorations.text,backgrounds,positioning,fit,petri,3d,calc}
%% 3d hnitakerfi
\usepackage{tikz-3dplot}
\tdplotsetmaincoords{75}{130}
%\begin{tikzpicture}[scale=5,tdplot_main_coords]
%
%%set up some coordinates 
%%-----------------------
%\coordinate (O) at (0,0,0);
%
%%determine a coordinate (P) using (r,\theta,\phi) coordinates.  This command
%%also determines (Pxy), (Pxz), and (Pyz): the xy-, xz-, and yz-projections
%%of the point (P).
%%syntax: \tdplotsetcoord{Coordinate name without parentheses}{r}{\theta}{\phi}
%\tdplotsetcoord{P}{\rvec}{\thetavec}{\phivec}


%%%%%%%%%%%%%%% SKIPANIR %%%%%%%%%%%%%%%%%%%%%%%%
%% LITASTYTTINGAR
%
\definecolor{dgreen}{rgb}{0,0.8,0}
\newcommand{\red}[1]{{\color{red} #1}}
\newcommand{\green}[1]{{\color{dgreen} #1}}
\newcommand{\blue}[1]{{\color{blue} #1}}
\newcommand{\black}[1]{{\color{black} #1}}
%
% ALLS KYNS STÆRÐFRÆÐIDÓT
%
\newcommand{\lvec}{\overrightarrow}
%
% ýmsar diffurstyttingar
\newcommand{\dif}{\,\mathrm{d}}
\newcommand{\dx}{\,\mathrm{d}x}
\newcommand{\dy}{\,\mathrm{d}y}
\newcommand{\dz}{\,\mathrm{d}z}
\newcommand{\dt}{\,\mathrm{d}t}
% afleiður og hlutafleiður
\renewcommand{\d}[2]{\frac{\dif{#1}}{\dif{#2}}}
\newcommand{\dd}[3]{\frac{\dif^{#1}{#2}}{\dif^{#1}{#3}}}
\newcommand{\p}[2]{\frac{\partial{#1}}{\partial {#2}}}
\newcommand{\pp}[3]{\frac{\partial^{#1}{#2}}{\partial{#3}^{#1}}}
%
% bil, millistig \; og \quad
\newcommand{\bil}{\hspace*{6pt}}
\newcommand{\Bil}{\hspace*{10pt}}
\newcommand{\vbil}{\vspace*{-6pt}}
\newcommand{\vBil}{\vspace*{-10pt}}

% samasemmerki með aukaplássi á báða bóga
\newcommand{\bils}{\bil=\bil}
%
\newcommand{\bc}{\begin{center}}
\newcommand{\ec}{\end{center}}
%
\newcommand{\beq}{\begin{equation}}
\newcommand{\eeq}{\end{equation}}
%
\newcommand{\beqa}{\begin{eqnarray*}}
\newcommand{\eeqa}{\end{eqnarray*}}
%
\newcommand{\ba}{\begin{array}}
\newcommand{\ea}{\end{array}}
%
\newcommand{\bma}{\begin{matrix}}
\newcommand{\ema}{\end{matrix}}
%
\newcommand{\bmh}{\left[\begin{matrix}}
\newcommand{\emh}{\end{matrix}\right]}
%
\newcommand{\ts}{\textstyle}
\newcommand{\ds}{\displaystyle}

% Frá mér (Hannesi)
\newcommand{\lausn}{\textbf{\textit{Lausn:}}}
\newcommand{\solution}{\textbf{\textit{Solution:}}}
\newcommand{\daemi}{\textbf{\textit{Dæmi:}}}
\newcommand{\bt}{\begin{table}[H]}
\newcommand{\et}{\end{table}}
\newcommand{\atm}{\text{atm}}
\newcommand{\calories}{\text{cal}}
\everymath{\displaystyle} % Svo allar jöfnur taka aldrei minna pláss
\renewcommand\tabcolsep{2pt} % Bil dalka í töflum
\setlength{\arraycolsep}{1.5pt}
%\usepackage{fancyhdr}
%\setlength{\headheight}{25pt}
%\pagestyle{fancy}
%\renewcommand{\headrulewidth}{0.4pt}
%\renewcommand{\footrulewidth}{0.4pt}
%\usepackage[top=2.5in, bottom=1.5in, left=1in, right=1in]{geometry}
\usepackage{amssymb}
\usepackage{pifont}
\newcommand{\cmark}{\text{\ding{51}}}%
\newcommand{\xmark}{\text{\ding{55}}}%

\usepackage{pbox}
%\usepackage{fancyvrb}
\usepackage{array}
\newcolumntype{L}[1]{>{\raggedright\let\newline\\\arraybackslash\hspace{0pt}}m{#1}}
\newcolumntype{C}[1]{>{\centering\let\newline\\\arraybackslash\hspace{0pt}}m{#1}}
\newcolumntype{R}[1]{>{\raggedleft\let\newline\\\arraybackslash\hspace{0pt}}m{#1}}
\usepackage[utf8]{inputenc}
\usepackage{framed}
\usepackage{paralist}
\renewenvironment{enumerate}[1]{\begin{compactenum}#1}{\end{compactenum}}
\usetikzlibrary{shapes.multipart,positioning,arrows}
\definecolor{pblue}{rgb}{0.13,0.13,1}
\definecolor{pgreen}{rgb}{0,0.5,0}
\definecolor{pred}{rgb}{0.9,0,0}
\definecolor{pgrey}{rgb}{0.46,0.45,0.48}
\usepackage{listings}
\lstset{language=Java,
  tabsize=4,
  numbers=left, 
  numberstyle=\small, 
  numbersep=8pt, 
  frame = single,
  showspaces=false,
  showtabs=false,
  breaklines=true,
  showstringspaces=false,
  breakatwhitespace=true,
  commentstyle=\color{pgreen},
  keywordstyle=\color{pblue},
  stringstyle=\color{pred},
  basicstyle=\ttfamily,
  moredelim=[il][\textcolor{pgrey}]{$$}, %$
  moredelim=[is][\textcolor{pgrey}]{\%\%}{\%\%}
} 

\tikzset{umlclass/.style={
        draw=black,fill=yellow!16,rectangle split,align=center, rectangle split part align={center,left}, minimum width=4cm,rounded corners},draw,rectangle split parts=4}

\begin{document}
\begin{titlepage}
\begin{center}
\textsc{}\\[2cm] 

\includegraphics[width=6cm]{Haskoli_Islands_rett.jpg}\\[0.5cm]

\HRule \\[0.6cm]
{ \huge \bfseries Group assignment 3: Tests}\\[0.2cm]
\HRule \\[0.4cm]

\textsc{\normalsize Þróun hugbúnaðar} \\
\textsc{Spring 2015} \\[1.5cm]

\begin{minipage}{0.45\textwidth}
\begin{flushleft} \large
\textit{Students:} (Group F2a)\\
\textsc{Einar Helgi Þrastarson} \\
\textsc{Hannes Pétur Eggertsson} \\
\textsc{Sigurður Birkir Sigurðsson} \\
\end{flushleft}
\end{minipage}
\begin{minipage}{0.45\textwidth}
\begin{flushright} \large
\textit{Teachers:} \\
\textsc{Matthias Book}\\
\textsc{Kristín Fjóla Tómasdóttir}\\
\textsc{ }\\
\end{flushright}
\end{minipage}

\end{center}
\end{titlepage}


% Description of the project
\section{Introduction}
In this document there's information about the tests for group F2a. Group members are: Einar Helgi Þrastarson (personal ID number: 110287-2919), Hannes Pétur Eggertsson (240889-2939) and Sigurður Birkir Sigurðsson (120589-2539). Our project is to build an user interface for a fantasy football game.

The presenter on Wednesday, March 11th 2015, will be Sigurður Birkir Sigurðsson.

\section{Test fixtures}
We created test fixture using JUnit to constantly test our core functionality. The core functionality we decided to test some of functionality on the roster class and the user class.

\subsection{Roster test fixtures}
We have made some small changes to the class and currently it looks as such:

\begin{figure}[H]
\centering
\begin{tikzpicture}
\node[umlclass] (roster)
 {\textbf{\large \textit{Roster}}
 \nodepart{two}
  {\footnotesize
  \begin{tabular}{l} 
   Keeps track of which football players are in which user team/roster.
  \end{tabular}}
 \nodepart{three}
  \begin{tabular}{l}
   -- \texttt{Player[2] goalkeepers}\\
   -- \texttt{Player[1] goalkeepersOnField}\\
   -- \texttt{Player[5] defenders}\\
   -- \texttt{Player[5] defendersOnField}\\
   -- \texttt{Player[5] midfielders}\\
   -- \texttt{Player[5] midfieldersOnField}\\
   -- \texttt{Player[3] strikers}\\
   -- \texttt{Player[3] strikersOnField}\\
   -- \texttt{Player captain}\\
   -- \texttt{int numberOfPlayersOnField}\\
  \end{tabular}
 \nodepart{four}
  \begin{tabular}{l}
   + \texttt{Roster()}\\
   + \texttt{void removePlayerFromTeam(Player player)}\\
   + \texttt{boolean addPlayerToTeam(Player player)}\\
   + \texttt{void changeCaptain(Player newCaptain)}\\
   + \texttt{boolean putPlayerOnField(Player player)}\\
   + \texttt{boolean removePlayerFromField(Player player)}\\
  \end{tabular}
 };
\end{tikzpicture}
\caption{The modified Roster class}
\end{figure}
The class has been created and will be the target for our test fixtures. The full source code of the class can be seen in the appendix.

\subsubsection{Set up and tear down}
The source code of our set up and tear down is long and boring. To make a long story short we created multiple PlayerMock (more information on those in the next section) objects on the form:
\begin{verbatim}
position1 = new PlayerMock("Position 1","Position");
\end{verbatim}
where position is a position of the player, e.g. goalkeeper or a midfielder. There is also one different PlayerMock object:
\begin{verbatim}
invalid_pos1 = new PlayerMock("Football fan","Couch potato");
\end{verbatim}
This is test how the class will handle a player of invalid position. After all players have been created they are all added to a hash-table \texttt{players} with their name a the key. If you wish to view the full source code in can be found in the appendix.

\subsubsection{Comparison of lists containing lists}
To help us in the tests it made sense to create a function that will compare two lists containing lists. One containing lists of lists with strings with the expected player names and one lists of lists containing a class of type PlayerInterface (an interface PlayerMock implemented).

\begin{lstlisting}
// Usage: m = compareListsOfLists(expected,actual)
// Before:expected is a list of lists with playernames in a String and actual
//        is a list of lists with players of type PlayerInterface.
//        From PlayerInterface we can retrieve the player's name with the
//        getName() method.
// After: If the list of lists contain the same players m will be the number of
//        players were matched. If the lists do not contain the same players
//        m will be returned as -1.
public int compareListsOfLists(List<List<String>> expected, List<List<PlayerInterface>> actual) throws IllegalStateException{
	// Count the number of matches
	int matches = 0;
	// If the two lists (of lists) have different number of elements, throw exception.
	if (actual.size() != expected.size()){
		throw new IllegalStateException("Sizes of lists containing lists not the same: "+expected.size()+" and "+actual.size());
	}
	// Create an iterator for both lists (of lists)
	Iterator<List<PlayerInterface>> playerlist_iterator = actual.iterator();
	Iterator<List<String>> expected_playerlist_iterator = expected.iterator();
	// Loop through the outer lists
	while(playerlist_iterator.hasNext()){
		List<PlayerInterface> playerlist = playerlist_iterator.next();
		List<String> expected_playerlist = expected_playerlist_iterator.next();
		
		if (playerlist.size() != expected_playerlist.size()){
			throw new IllegalStateException("Sizes of lists not the same: "+expected_playerlist.size()+" and "+playerlist.size());
		}
		
		Iterator<PlayerInterface> player_iterator = playerlist.iterator();
		Iterator<String> expected_player_iterator = expected_playerlist.iterator();
		
		// Loop through the inner lists
		while(player_iterator.hasNext()){
			String expected_player = expected_player_iterator.next();
			if(player_iterator.next().getName() != expected_player){
				return -1;
			}
			else{
				matches++;
			}
		}
	}
	return matches;
}
\end{lstlisting}

\subsubsection{Testing the addPlayerToRoster method}
\subsubsection*{Test 1: List, art thou empty?}
This test will check if a new roster is empty.
\begin{lstlisting}
public void testIfEmpty() throws IllegalStateException, InvalidPosition {
	List<List<PlayerInterface>> actual = roster.getPlayersInRoster();
	List<List<String>> excepted = new ArrayList<List<String>>(4) {{add(goalkeepers);add(defenders);add(midfielders);add(strikers);}};
	assertEquals(0,compareListsOfLists(excepted, actual));
}
\end{lstlisting}

\subsubsection*{Test 2: Takes one to know one}
This test will check if we can successfully add a single player to the roster.
\begin{lstlisting}
public void testIfOnePlayer() throws IllegalStateException, InvalidPosition {
	// Add the player "Goalkeeper 1" to the roster
	boolean add = roster.addPlayerToRoster(players.get("Goalkeeper 1"));
	assertTrue(add);
	
	// Get the roster players
	List<List<PlayerInterface>> actual = roster.getPlayersInRoster();
	
	// Create the expected outcome of the test
	goalkeepers.add("Goalkeeper 1");
	List<List<String>> excepted = new ArrayList<List<String>>(4) {{add(goalkeepers);add(defenders);add(midfielders);add(strikers);}};
	
	assertEquals(1,compareListsOfLists(excepted, actual));
}
\end{lstlisting}

\subsubsection*{Test 3: Who invited you?}
This test will check if we get an exception when adding a player with a invalid position. We expect to get a \texttt{InvalidPosition} exception in that case.
\begin{lstlisting}
public void testIfInvalidPlayer() throws InvalidPosition {
	Throwable exception = null;
	// Add the player "Football fan" to the roster
	try{
		roster.addPlayerToRoster(players.get("Football fan"));
	} catch (Throwable e) {
		exception = e;
	}
	assertNotNull(exception);
	assertSame(InvalidPosition.class,exception.getClass());
}
\end{lstlisting}

\subsubsection*{Test 4: Only two can tango}
This test will check if we will receive "false" from the \texttt{addPlayerToRoster()} method if we try to add too many players to the same position.
\begin{lstlisting}
public void testIfThreePlayers() throws InvalidPosition {
	// Add the player "Goalkeeper 1" to the roster
	roster.addPlayerToRoster(players.get("Goalkeeper 1"));
	boolean add = roster.addPlayerToRoster(players.get("Goalkeeper 2"));
	assertTrue(add);
	add = roster.addPlayerToRoster(players.get("Goalkeeper 3"));
	assertFalse(add);
	
	// Get the roster players
	List<List<PlayerInterface>> actual = roster.getPlayersInRoster();
	
	// Create the expected outcome of the test
	goalkeepers.add("Goalkeeper 1");
	goalkeepers.add("Goalkeeper 2");
	List<List<String>> excepted = new ArrayList<List<String>>(4) {{add(goalkeepers);add(defenders);add(midfielders);add(strikers);}};
	
	assertEquals(2,compareListsOfLists(excepted, actual));
}
\end{lstlisting}

\subsubsection*{Test 5: No more room in heaven}
\begin{lstlisting}
public void testIfFullRoster() throws InvalidPosition {
	// Add the player "Goalkeeper 1" to the roster
	roster.addPlayerToRoster(players.get("Goalkeeper 1"));
	roster.addPlayerToRoster(players.get("Goalkeeper 2"));
	boolean add;
	add = roster.addPlayerToRoster(players.get("Defender 1"));	assertTrue(add);
	add = roster.addPlayerToRoster(players.get("Defender 2"));	assertTrue(add);
	add = roster.addPlayerToRoster(players.get("Defender 3"));	assertTrue(add);
	add = roster.addPlayerToRoster(players.get("Defender 4"));	assertTrue(add);
	add = roster.addPlayerToRoster(players.get("Defender 5"));	assertTrue(add);
	add = roster.addPlayerToRoster(players.get("Midfielder 1"));	assertTrue(add);
	add = roster.addPlayerToRoster(players.get("Midfielder 2"));	assertTrue(add);
	add = roster.addPlayerToRoster(players.get("Midfielder 3"));	assertTrue(add);
	add = roster.addPlayerToRoster(players.get("Midfielder 4"));	assertTrue(add);
	add = roster.addPlayerToRoster(players.get("Midfielder 5"));	assertTrue(add);
	add = roster.addPlayerToRoster(players.get("Striker 1"));		assertTrue(add);
	add = roster.addPlayerToRoster(players.get("Striker 2"));		assertTrue(add);
	add = roster.addPlayerToRoster(players.get("Striker 3"));		assertTrue(add);
	
	// Get the roster players
	List<List<PlayerInterface>> actual = roster.getPlayersInRoster();
	
	// Create the expected outcome of the test
	goalkeepers.add("Goalkeeper 1");
	goalkeepers.add("Goalkeeper 2");
	defenders.add("Defender 1");
	defenders.add("Defender 2");
	defenders.add("Defender 3");
	defenders.add("Defender 4");
	defenders.add("Defender 5");
	midfielders.add("Midfielder 1");
	midfielders.add("Midfielder 2");
	midfielders.add("Midfielder 3");
	midfielders.add("Midfielder 4");
	midfielders.add("Midfielder 5");
	strikers.add("Striker 1");
	strikers.add("Striker 2");
	strikers.add("Striker 3");
	List<List<String>> excepted = new ArrayList<List<String>>(4) {{add(goalkeepers);add(defenders);add(midfielders);add(strikers);}};
	
	assertEquals(15,compareListsOfLists(excepted, actual));
}
\end{lstlisting}
\subsubsection{Testing the addPlayerToField method}
\subsubsection*{Test 6: We must follow the rules}
This test will check if we can add too many players to the same position
\begin{lstlisting}
public void testIfAddGoalkeepers() throws InvalidPlayer, InvalidPosition {
	roster.addPlayerToRoster(players.get("Goalkeeper 1"));
	roster.addPlayerToRoster(players.get("Goalkeeper 2"));
	boolean b = roster.addPlayerToField(players.get("Goalkeeper 1"));
	assertTrue(b);
	b = roster.addPlayerToField(players.get("Goalkeeper 2"));
	assertFalse(b);
}
\end{lstlisting}

\subsubsection*{Test 7: You can't play with us}
This test will check if we can successfully add eleven players to the field and can't add the twelfth.
\begin{lstlisting}
public void testIfAddElevenAndTwelveToField() throws InvalidPlayer, InvalidPosition {
	// All 15 test players available in roster
	roster.addPlayerToRoster(players.get("Goalkeeper 1"));
	roster.addPlayerToRoster(players.get("Goalkeeper 2"));
	roster.addPlayerToRoster(players.get("Defender 1"));
	roster.addPlayerToRoster(players.get("Defender 2"));
	roster.addPlayerToRoster(players.get("Defender 3"));
	roster.addPlayerToRoster(players.get("Defender 4"));
	roster.addPlayerToRoster(players.get("Defender 5"));
	roster.addPlayerToRoster(players.get("Midfielder 1"));
	roster.addPlayerToRoster(players.get("Midfielder 2"));
	roster.addPlayerToRoster(players.get("Midfielder 3"));
	roster.addPlayerToRoster(players.get("Midfielder 4"));
	roster.addPlayerToRoster(players.get("Midfielder 5"));
	roster.addPlayerToRoster(players.get("Striker 1"));
	roster.addPlayerToRoster(players.get("Striker 2"));
	roster.addPlayerToRoster(players.get("Striker 3"));
	
	boolean b;
	roster.addPlayerToField(players.get("Goalkeeper 1"));	
	b = roster.addPlayerToField(players.get("Defender 1"));		assertTrue(b);
	b = roster.addPlayerToField(players.get("Defender 2"));		assertTrue(b);
	b = roster.addPlayerToField(players.get("Midfielder 1"));	assertTrue(b);
	b = roster.addPlayerToField(players.get("Midfielder 2"));	assertTrue(b);
	b = roster.addPlayerToField(players.get("Midfielder 3"));	assertTrue(b);
	b = roster.addPlayerToField(players.get("Midfielder 4"));	assertTrue(b);
	b = roster.addPlayerToField(players.get("Midfielder 5"));	assertTrue(b);
	b = roster.addPlayerToField(players.get("Striker 1"));		assertTrue(b);
	b = roster.addPlayerToField(players.get("Striker 2"));		assertTrue(b);
	b = roster.addPlayerToField(players.get("Striker 3"));		assertTrue(b);
	
	// Test if adding a player that is not in the roster will throw the InvalidPlayer exception
	Throwable exception = null;
	try{
		roster.addPlayerToField(players.get("Football fan"));
	} catch (Throwable e) {
		exception = e;
	}
	assertNotNull(exception);
	assertSame(InvalidPlayer.class,exception.getClass());
	
	// Test if we're not able to add the 12th player to the field
	b = roster.addPlayerToField(players.get("Defender 3"));
	assertFalse(b);
}
\end{lstlisting}

\subsubsection{Testing the removeFromRoster method}
\subsubsection*{Test 8: Join us! Now leave us!}
Test if we remove a player that's in the roster.
\begin{lstlisting}
public void testRemoveFromRoster() throws InvalidPosition, InvalidPlayer{
	roster.addPlayerToRoster(players.get("Goalkeeper 1"));
	roster.removePlayer(players.get("Goalkeeper 1"), true);
	
	// Check if the roster is empty
	List<List<PlayerInterface>> actual = roster.getPlayersInRoster();
	List<List<String>> excepted = new ArrayList<List<String>>(4) {{add(goalkeepers);add(defenders);add(midfielders);add(strikers);}};
	assertEquals(0,compareListsOfLists(excepted, actual));
}
\end{lstlisting}

\subsubsection*{Test 9: Go away nobody}
Test if we remove a player that's NOT in the roster. We expect it to thrown an exception.
\begin{lstlisting}
public void testRemoveInvalidPlayer() {
	Throwable exception = null;
	try{
		roster.removePlayer(players.get("Goalkeeper 1"), true);
	} catch (Throwable e) {
		exception = e;
	}
	assertNotNull(exception);
	assertSame(InvalidPlayer.class,exception.getClass());
}
\end{lstlisting}

\subsubsection*{Test 10: 1,2,3,...}
This test if check if the variable NumberOfPlayersOfField is changed correctly.
\begin{lstlisting}
public void testNumberOfPlayersOnField() throws InvalidPosition, InvalidPlayer {
	assertEquals(0,roster.getNumberOfPlayersOnField());
	roster.addPlayerToRoster(players.get("Goalkeeper 1"));
	roster.addPlayerToRoster(players.get("Defender 2"));
	assertEquals(0,roster.getNumberOfPlayersOnField());
	roster.addPlayerToField(players.get("Goalkeeper 1"));
	assertEquals(1,roster.getNumberOfPlayersOnField());
	roster.addPlayerToField(players.get("Defender 2"));
	assertEquals(2,roster.getNumberOfPlayersOnField());
	roster.removePlayer(players.get("Goalkeeper 1"), false);
	assertEquals(1,roster.getNumberOfPlayersOnField());
	roster.removePlayer(players.get("Defender 2"), true);
	assertEquals(0,roster.getNumberOfPlayersOnField());
}
\end{lstlisting}

\subsection{User test fixtures}
\subsubsection{Set up and tear down}
This test, tests the class that holds onto the users of the game. We setup the test with one new user which takes in "user1" as its name.

\begin{lstlisting}
@BeforeClass
public static void setUp() throws Exception {
	user = new User("user1");
}
\end{lstlisting}

\begin{lstlisting}
@AfterClass
public static void tearDown() throws Exception {
	user = null;
}
\end{lstlisting}

\subsubsection{Testing the user classes}
\subsubsection*{Test 1: Remember kids, null != empty}
This test checks if the user has a Roster.
\begin{lstlisting}
@Test
public void checkRoster() {
	assertNotNull(user.getRoster());
}

\end{lstlisting}
\subsubsection*{Test 2: Mannanafnanefnd}
This test checks if a name change goes though.
\begin{lstlisting}
@Test 
public void nameChange() {
	user.changeName("user2");
	assertEquals("user2", user.getName());
}
\end{lstlisting}
\subsubsection*{Test 3: Make sure to put it in the bank}
\begin{lstlisting}
@Test
public void testChangingMoney() {
	
	// Money is 0
	assertTrue(user.changeMoney(1000));
	assertEquals(1000, user.getMoney());
	
	// Money is 1000
	int curr = user.getMoney();
	assertTrue(user.changeMoney(-900));
	assertEquals(curr-900, user.getMoney());
	
	// Money is 100
	curr = user.getMoney();
	assertFalse(user.changeMoney(-200));
	assertEquals(curr, user.getMoney());
	// Money is still 100 cause you can't get negative money
}
\end{lstlisting}

\newpage
\section{Mock objects}
In order to have the test fixtures above we needed to create a mock up class for the player, we call PlayerMockup. Into this class we put the most basic information about the player and didn't create any unnecessary methods the real Player class will have when it's created by group F1a.

\begin{lstlisting}
public class PlayerMock implements PlayerInterface {

	private String name;
	private PositionMock position;
	private String positionName;
	
	public PlayerMock(String name, String pos){
		this.name = name;
		this.positionName = pos;
	}
	
	public String getName(){
		return this.name;
	}
	
	@Override
	public String getPositionName() {
		return this.positionName;
	}
	
	@Override
	public void setPosition(PositionMock pos) throws InvalidPosition{
		if (pos.equals("Goalkeeper") || pos.equals("Defender") || pos.equals("Midfielder") || pos.equals("Striker")){
			this.position = pos;
		} else {
			throw new InvalidPosition(pos+" is not a valid position. Only Goalkeeper, Defender, Midfielder, and Striker are valid.");
		}
	}
	
	@Override
	public PositionMock getPosition() {
		return position;
	}
	
}
\end{lstlisting}
This class implements the PlayerInterface
\begin{lstlisting}
public interface PlayerInterface {
	public String getName();
	public void setPosition(PositionMock pos) throws InvalidPosition;
	public PositionMock getPosition();
	public String getPositionName();
}
\end{lstlisting}
The class is made so it will work successfully on the test fixtures but doesn't include any other information or methods not used.

\section{Test cases}
We decided to create test cases for the search feature on the market panel. The goal of the feature is to provide a accurate search of all players and possible have some useful filters. The filters we chose to include in these test cases were "Teams", and "Position", i.e. you can choose a team and/or position to filter out players. The description of the function:
\begin{lstlisting}
// Usage: matches = searchPlayers(String search_term, Team filtered_team, 
//        Position filtered_position)
// Before:search_term is the term currently being searched for, filtered_team
//        is the filtered team (can be null if not specified), and
//        filtered_position is a position for a player (can also be null if
//        not specified).
// After: Search results for the the search_term using (if not null) the two
//        filters. Match is when the search_term is a substring of the 
//        player's name. All matches are displayed on the screen.
\end{lstlisting}

A table of some test cases we would use to test this feature would be:
\begin{table}[H]
\begin{tabular}{|l|l|l|l|l|ll}
\cline{1-5}
\multicolumn{3}{|c|}{\textbf{Before}} & \multicolumn{2}{c|}{\textbf{After}}\\ \cline{1-5}
\textbf{Search term} & \textbf{Team}  & \textbf{Position}   & \textbf{Expected results} & \textbf{Explanation} &  &\\ \cline{1-5}
""           & null            & null       & Aaron Lennon, Aaron Ramsey,...             & All players                                      &  &  \\ \cline{1-5}
Rooney       & null            & null       & Wayne Rooney, John Rooney,... & All players with Rooney as a subsrting         &  &  \\ \cline{1-5}
Rooney       & null            & Striker    & Wayne Rooney, Adam Rooney,...              & All strikers having the Rooney subsrtring      &  &  \\ \cline{1-5}
Rooney       & null            & Goalkeeper & ""                                           & All Rooney's in that are Goalkeepers             &  &  \\ \cline{1-5}
Rooney       & Chelsea         & null       & ""                                           & All Rooney's in Chelsea                          &  &  \\ \cline{1-5}
Rooney       & Man. Utd. & null       & Wayne Rooney                               & All Rooney's in Man. Utd.                  &  &  \\ \cline{1-5}
Rooney       & Man. Utd. & Striker    & Wayne Rooney                               & All Rooney's in Man. Utd. and are strikers &  &  \\ \cline{1-5}
Wayne Rooney & null            & null       & Wayne Rooney                               & All players with the Wayne Rooney substring   &  &  \\ \cline{1-5}
\end{tabular}
\end{table}


\newpage
\section*{Appendix}
\subsection*{RosterTest.java}
\subsubsection*{Set up before the class method}
\begin{lstlisting}
public static void setUpBeforeClass() {
	goalkeeper1 = new PlayerMock("Goalkeeper 1","Goalkeeper");
	goalkeeper2 = new PlayerMock("Goalkeeper 2","Goalkeeper");
	goalkeeper3 = new PlayerMock("Goalkeeper 3","Goalkeeper");
	
	invalid_pos1 = new PlayerMock("Football fan","Couch potato");
	
	defender1 = new PlayerMock("Defender 1","Defender");
	defender2 = new PlayerMock("Defender 2","Defender");
	defender3 = new PlayerMock("Defender 3","Defender");
	defender4 = new PlayerMock("Defender 4","Defender");
	defender5 = new PlayerMock("Defender 5","Defender");
	
	midfielder1 = new PlayerMock("Midfielder 1","Midfielder");
	midfielder2 = new PlayerMock("Midfielder 2","Midfielder");
	midfielder3 = new PlayerMock("Midfielder 3","Midfielder");
	midfielder4 = new PlayerMock("Midfielder 4","Midfielder");
	midfielder5 = new PlayerMock("Midfielder 5","Midfielder");
	
	striker1 = new PlayerMock("Striker 1","Striker");
	striker2 = new PlayerMock("Striker 2","Striker");
	striker3 = new PlayerMock("Striker 3","Striker");
	
	players = new HashMap<String, PlayerMock>();
	players.put(goalkeeper1.getName(),goalkeeper1);
	players.put(goalkeeper2.getName(),goalkeeper2);
	players.put(goalkeeper3.getName(),goalkeeper3);
	
	players.put(invalid_pos1.getName(),invalid_pos1);
	
	players.put(defender1.getName(),defender1);
	players.put(defender2.getName(),defender2);
	players.put(defender3.getName(),defender3);
	players.put(defender4.getName(),defender4);
	players.put(defender5.getName(),defender5);
	
	players.put(midfielder1.getName(),midfielder1);
	players.put(midfielder2.getName(),midfielder2);
	players.put(midfielder3.getName(),midfielder3);
	players.put(midfielder4.getName(),midfielder4);
	players.put(midfielder5.getName(),midfielder5);
	
	players.put(striker1.getName(), striker1);
	players.put(striker2.getName(), striker2);
	players.put(striker3.getName(), striker3);
}
\end{lstlisting}

\subsubsection*{Before each test method}
\begin{lstlisting}
@Before
public void setUp() throws Exception {
	roster = new Roster();
	goalkeepers = new ArrayList<String>(2);
	defenders = new ArrayList<String>(5);
	midfielders = new ArrayList<String>(5);
	strikers = new ArrayList<String>(3);
}
\end{lstlisting}

\subsubsection*{After each test method}
\begin{lstlisting}
@After
public void tearDown() throws Exception {
	roster = null;
	goalkeepers = null;
	defenders = null;
	midfielders = null;
	strikers = null;
}
\end{lstlisting}

\subsection*{Roster.java}
\begin{lstlisting}
import java.util.ArrayList;
import java.util.List;

import tests.*;

public class Roster {
	private List<PlayerInterface> goalkeepers;
	private List<PlayerInterface> goalkeeperOnField;
	private List<PlayerInterface> defenders;
	private List<PlayerInterface> defendersOnField;
	private List<PlayerInterface> midfielders;
	private List<PlayerInterface> midfieldersOnField;
	private List<PlayerInterface> strikers;
	private List<PlayerInterface> strikersOnField;
	// private PlayerInterface captain;
	private int numberOfPlayersOnField;
	
	public Roster(){
		this.numberOfPlayersOnField = 0;
		this.goalkeepers = new ArrayList<PlayerInterface>(2);
		this.goalkeeperOnField = new ArrayList<PlayerInterface>(1);
		this.defenders = new ArrayList<PlayerInterface>(5);
		this.defendersOnField = new ArrayList<PlayerInterface>(5);
		this.midfielders = new ArrayList<PlayerInterface>(5);
		this.midfieldersOnField = new ArrayList<PlayerInterface>(5);
		this.strikers = new ArrayList<PlayerInterface>(3);
		this.strikersOnField = new ArrayList<PlayerInterface>(3);
	}
	
	// Usage: i = getNumberOfPlayersOnField()
	// Before:Nothing.
	// After: i is the number of players currently on the field.
	public int getNumberOfPlayersOnField(){
		return this.numberOfPlayersOnField;
	}
	
	// Usage: removePlayer(player,b)
	// Before:player is of type PlayerInterface and b is a boolean variable (true or false)
	// After: If b is true then player will be removed both from the field and the roster. If
	//        b is false then the player will only be removed from the field. If the player
	//        provided is not in the roster then a InvalidPlayer exception will be thrown.
	public void removePlayer(PlayerInterface player, boolean removeFromRoster) throws InvalidPlayer{
		String posName = player.getPositionName();
		if (posName.toLowerCase().equals("goalkeeper")){
			if (removeFromRoster){
				boolean b = goalkeepers.remove(player);
				if (!b) throw new tests.InvalidPlayer("The player "+player.getName()+" isn't in this roster");
			}
			goalkeeperOnField.remove(player);
			numberOfPlayersOnField--;
		} else if (posName.toLowerCase().equals("defender")){
			if (removeFromRoster){
				boolean b = defenders.remove(player);
				if (!b) throw new tests.InvalidPlayer("The player "+player.getName()+" isn't in this roster");
			}
			defendersOnField.remove(player);
			numberOfPlayersOnField--;
		} else if (posName.toLowerCase().equals("midfielder")){
			if (removeFromRoster){
				boolean b = midfielders.remove(player);
				if (!b) throw new tests.InvalidPlayer("The player "+player.getName()+" isn't in this roster");
			}
			midfieldersOnField.remove(player);
			numberOfPlayersOnField--;
		} else if (posName.toLowerCase().equals("striker")){
			if (removeFromRoster){
				boolean b = strikers.remove(player);
				if (!b) throw new tests.InvalidPlayer("The player "+player.getName()+" isn't in this roster");
			}
			strikersOnField.remove(player);
			numberOfPlayersOnField--;
		}
	}
	
	// Usage: b = addPlayerToRoster(player)
	// Before:player is of type PlayerInferface and is not null
	// After: If there is room for the player in roster in the position that he plays then he's
	//        added to the roster and b is returned as true. If there is no room him in his
	//        position then b is returned as false. If the player's position is not "Goalkeeper",
	//        "Defender", "Midfielder", or "Striker" then InvalidPosition exception is thrown.
	public boolean addPlayerToRoster(PlayerInterface player) throws InvalidPosition{
		String posName = player.getPositionName();
		if (posName.toLowerCase().equals("goalkeeper")){
			if (this.goalkeepers.size() == 2) return false;
			this.goalkeepers.add(player);
			return true;
		} else if (posName.toLowerCase().equals("defender")){
			if (this.defenders.size() == 5) return false;
			this.defenders.add(player);
			return true;
		} else if (posName.toLowerCase().equals("midfielder")){
			if (midfielders.size() == 5) return false;
			this.midfielders.add(player);
			return true;
		} else if (posName.toLowerCase().equals("striker")){
			if (strikers.size() == 3) return false;
			this.strikers.add(player);
			return true;
		} else {
			throw new InvalidPosition(posName+" is not a valid position. Only Goalkeeper, Defender, Midfielder, and Striker are valid.");
		}
	}
	
	// Usage: b = addPlayerToField(player)
	// Before:player is of type PlayerInterface
	// After: If the player isn't in the roster then an InvalidPlayer exception is thrown. Otherwise, and
	//        if the player's position and roster on field are not full, the player will be added to the field
	//        and b will be returned as true. Otherwise b will be returned as false.
	public boolean addPlayerToField(PlayerInterface player) throws InvalidPlayer{
		if (this.goalkeepers.contains(player)){
			if (this.goalkeeperOnField.size() >= 1 || this.numberOfPlayersOnField >= 11){
				return false;
			}
			this.goalkeeperOnField.add(player);
			this.numberOfPlayersOnField++;
			return true;
		} else if (this.defenders.contains(player)){
			if (this.defendersOnField.contains(player) || this.numberOfPlayersOnField >= 11){
				return false;
			}
			this.defendersOnField.add(player);
			this.numberOfPlayersOnField++;
			return true;
		} else if (this.midfielders.contains(player)){
			if (this.midfieldersOnField.contains(player) || this.numberOfPlayersOnField >= 11){
				return false;
			}
			this.midfieldersOnField.add(player);
			this.numberOfPlayersOnField++;
			return true;
		} else if (this.strikers.contains(player)){
			if (this.strikersOnField.contains(player) || this.numberOfPlayersOnField >= 11){
				return false;
			}
			this.strikersOnField.add(player);
			this.numberOfPlayersOnField++;
			return true;
		} else {
			throw new tests.InvalidPlayer(player.getName()+" is currently not in the roster.");
		}
	}
	
	// Usage: getPlayersInRoster()
	// Before:Nothing
	// After: List containing a list of all players in the current roster. There will always be 4 inner
	//        lists, the first for goalkeepers, second for defenders, third for midfielders, and the
	//        4th for strikers.
	public List<List<PlayerInterface>> getPlayersInRoster(){
		List<List<PlayerInterface>> names = new ArrayList<List<PlayerInterface>>(4);
		names.add(goalkeepers);
		names.add(defenders);
		names.add(midfielders);
		names.add(strikers);
		return names;
	}
}
\end{lstlisting}

\newpage\subsection*{User.java}
\begin{lstlisting}
public class User {
	private int money;
	private int[] score;
	private String name;
	private Roster roster;
	
	public User(String name) {
		this.name = name;
		this.score = new int[10];
		this.roster = new Roster();
		this.money = 0;
	}
	
	public boolean changeMoney(int dMoney) {
		if(this.money + dMoney <0) return false;
		this.money += dMoney;
		return true;
	}
	
	public int getMoney(){
		return this.money;
	}
	
	public Roster getRoster() {
		return this.roster;
	}

	// Usage: i = getScore()
	// Before:Nothing.
	// After: i is an array of scores of this user
	public int[] getScore() {
		return this.score;
	}
	
	public void addScore(int round, int score) {
		this.score[round] += score;
		addScoreToStats();
	}
	
	public void addScoreToStats() {
		
	}
	
	public void changeName(String newname) {
		this.name = newname;
	}
	
	public String getName() {
		return name;
	}
}
\end{lstlisting}

\newpage
\subsection*{UserTest.java}
\begin{lstlisting}
import static org.junit.Assert.*;

import org.junit.AfterClass;
import org.junit.BeforeClass;
import org.junit.Test;

import backend.User;

public class UserTest {
	
	private static User user;

	// This test, tests the class that holds onto the users of the game.
	// We setup the test with one new user which takes in "user1" as its name.
	
	@BeforeClass
	public static void setUp() throws Exception {
		user = new User("user1");
	}

	@AfterClass
	public static void tearDown() throws Exception {
		user = null;
	}
	
	// This test checks if the user has a Roster
	@Test
	public void checkRoster() {
		assertNotNull(user.getRoster());
	}
	
	// This test checks if a name change goes though
	@Test 
	public void nameChange() {
		user.changeName("user2");
		assertEquals("user2", user.getName());
	}
	
	// This test checks if the score from round 0 goes to index 0
	@Test
	public void scoreTest() {
		user.addScore(0, 100);
		assertEquals(100, user.getScore()[0]);
	}
	
	// This test checks if the changes to the money of the user behave as intended
	// where the user can't have negative money and no money is subtracted if that subtraction 
	// would set the users money below zero
	@Test
	public void testChangingMoney() {
		
		// Money is 0
		assertTrue(user.changeMoney(1000));
		assertEquals(1000, user.getMoney());
		
		// Money is 1000
		int curr = user.getMoney();
		assertTrue(user.changeMoney(-900));
		assertEquals(curr-900, user.getMoney());
		
		// Money is 100
		curr = user.getMoney();
		assertFalse(user.changeMoney(-200));
		assertEquals(curr, user.getMoney());
		// Money is still 100 cause you can't get negative money
	}
}
\end{lstlisting}
\end{document}